%-+- coding:UTF-8  -+-
%模版.tex
%作者:孙霖,华北电力大学信息1502
%历史记录
%2019/04/11         SunLIn          建立模板并边学习

\documentclass[UTF8,a4paper]{ctexrep}
\usepackage{geometry}
\usepackage{fancyhdr}
\usepackage{graphicx}
\usepackage{titlesec}
\usepackage{fontspec}
\usepackage[titles]{tocloft}
% 字体字号设置
% 小四, 1.5倍行距
\newcommand{\xiaosi}{\fontsize{12pt}{18pt}\selectfont}            
% 四号, 1.5 倍行距
\newcommand{\sihao}{\fontsize{14pt}{21pt}\selectfont}            
\setmainfont{Times New Roman}
\fontsize{12pt}{18pt}
\selectfont
%%文档整体设置
%%设置页眉页脚
\fancyhead[L,R]{}
\fancyhead[C]{华北电力大学本科毕业设计(论文)}
%%标题整体设置
%\titleformat{\chapter}{\centering}{第\,\thechapter\,章}{1em}{}
\titleformat{\section}{\raggedright\zihao{3}\heiti}{\thesection \quad}{0pt}{}
\titleformat{\subsection}{\raggedright\zihao{-3}\heiti}{\thesubsection \quad }{0pt}{}
%%纸张规格为A4 ;版面上空2.5cm,下空2cm,左空2.5 cm,右空2 cm(左装订)

\geometry{
  left=2.5cm,
  right=2cm,
  top=2.5cm, 
  bottom=2cm,
}

\begin{document}
%%插入封面 
%%/***************************************************************/
%%摘要页

%%黑体四号,摘要之间空4个字符
%页码计数器
\chapter*{摘\qquad 要}{\heiti\zihao{4}}
\addcontentsline{toc}{chapter}{摘\qquad 要}
\pagenumbering{Roman}
%%空一行宋体小四号
空一行宋体小四号

\noindent
\qquad 摘要内容,首行空2字符,字数400字左右

%%用宋体小四号,首行空2字符,字数400字左右。
%空一行宋体小四号
%%添加关键词关键词三个字左顶格,用宋体小四号加粗,3-5个关键词用宋体小四号,
%%词于词之间用逗号隔开,最后一个词不加任何标点符号。

\noindent
关键词:
\clearpage
%%/***************************************************************/
%%ABSTRACT

%Times New Roman字体、小四加粗
\chapter*{\textbf{ABSTRACT}}{\zihao{-4}\bf}
\addcontentsline{toc}{chapter}{ABSTRACT}
%空一行,Times New Roman字体、小四号
%Times New Roman字体、小四,首行空4个字母

\qquad  ABSTRACT With English

%关键词左顶格,Times New Roman字体、小四加粗,3-5个关键词Times New Roman字体、小四号,
%词于词之间用逗号隔开,最后一个词不加任何标点符号。
KEYWORDS:
\newpage
%%/***************************************************************/
%%目录
%%%%%给第一级标题加点
\renewcommand{\cftdot}{$\cdot$}
\renewcommand{\cftdotsep}{1.5}
\setlength{\cftbeforechapskip}{10pt}

\renewcommand{\cftchapleader}{\cftdotfill{\cftchapdotsep}}
\renewcommand{\cftchapdotsep}{\cftdotsep}
\makeatletter
\renewcommand{\numberline}[1]{%
\settowidth\@tempdimb{#1\hspace{0.5em}}%
\ifdim\@tempdima<\@tempdimb%
  \@tempdima=\@tempdimb%
\fi%
\hb@xt@\@tempdima{\@cftbsnum #1\@cftasnum\hfil}\@cftasnumb}
\makeatother
%%%%%给第一级标题加点
%黑体四号,目录两个字之间空4个字
\pagenumbering{arabic}
\renewcommand\contentsname{目\qquad 录}
\tableofcontents
%\tableofcontents{section}{\songti\zihao{-4}}
%\tableofcontents{subsection}{\songti\zihao{-4}}


%黑体小四号:摘要、ABSTRACT、一级标题
%宋体小四号:二级三级
\clearpage
\pagenumbering{arabic}

%%/***************************************************************/
%%正文

%%/***************************************************************/
%%第一章
\chapter{绪论}{\heiti\zihao{-2}}
\thispagestyle{fancy}

  \section{课题背景}
  \section{国内外研究现状}
    \subsection{国内研究现状}
    \subsection{国外研究现状}
  \section{本文研究内容和结构组织}
\clearpage
\chapter{相关技术}{\heiti\zihao{-2}}
\thispagestyle{fancy}
\clearpage
\chapter{总体设计}{\heiti\zihao{-2}}
\thispagestyle{fancy}
\clearpage
\pagenumbering{arabic}
\chapter*{参考文献}{\heiti\zihao{-2}}
\addcontentsline{toc}{chapter}{参考文献}
\thispagestyle{fancy}
\chapter*{附\qquad 录}{\heiti\zihao{-2}}
\addcontentsline{toc}{chapter}{附\qquad 录}
\thispagestyle{fancy}
论文的附录依次按附录A,附录B 等进行编号。附录内容的书写格式按毕业设计(论文)的正文规定格式书写。
\chapter*{致\qquad 谢}{\heiti\zihao{-2}}
\addcontentsline{toc}{chapter}{致\qquad 谢}
\thispagestyle{fancy}
对曾经给予本人顺利完成毕业设计(论文)而提供各类帮助、指导,以及协助完成该项研究工作的单位和个人表示感谢。


\end{document}